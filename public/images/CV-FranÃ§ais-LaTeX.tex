%% Use the "normalphoto" option if you want a normal photo instead of cropped to a circle
% \documentclass[10pt,a4paper,normalphoto]{altacv}

\documentclass[10pt,a4paper,ragged2e,withhyper]{altacv}
%% AltaCV uses the fontawesome5 and packages.
%% See http://texdoc.net/pkg/fontawesome5 for full list of symbols.

% Change the page layout if you need to
\geometry{left=1.25cm,right=1.25cm,top=1.5cm,bottom=1.5cm,columnsep=1.2cm}

% The paracol package lets you typeset columns of text in parallel
\usepackage{paracol}

\ifxetexorluatex
  % If using xelatex or lualatex:
  \setmainfont{Roboto Slab}
  \setsansfont{Lato}
  \renewcommand{\familydefault}{\sfdefault}
\else
  % If using pdflatex:
  \usepackage[rm]{roboto}
  \usepackage[defaultsans]{lato}
  % \usepackage{sourcesanspro}
  \renewcommand{\familydefault}{\sfdefault}
\fi

% Change the colours if you want to
\definecolor{SlateGrey}{HTML}{2E2E2E}
\definecolor{LightGrey}{HTML}{666666}
\definecolor{DarkPastelRed}{HTML}{450808}
\definecolor{PastelRed}{HTML}{8F0D0D}
\definecolor{GoldenEarth}{HTML}{E7D192}
\colorlet{name}{black}
\colorlet{tagline}{PastelRed}
\colorlet{heading}{DarkPastelRed}
\colorlet{headingrule}{GoldenEarth}
\colorlet{subheading}{PastelRed}
\colorlet{accent}{PastelRed}
\colorlet{emphasis}{SlateGrey}
\colorlet{body}{LightGrey}

% Change some fonts, if necessary
\renewcommand{\namefont}{\Huge\rmfamily\bfseries}
\renewcommand{\personalinfofont}{\footnotesize}
\renewcommand{\cvsectionfont}{\LARGE\rmfamily\bfseries}
\renewcommand{\cvsubsectionfont}{\large\bfseries}


% Change the bullets for itemize and rating marker
% for \cvskill if you want to
\renewcommand{\cvItemMarker}{{\small\textbullet}}
\renewcommand{\cvRatingMarker}{\faCircle}
% ...and the markers for the date/location for \cvevent
% \renewcommand{\cvDateMarker}{\faCalendar*[regular]}
% \renewcommand{\cvLocationMarker}{\faMapMarker*}


% If your CV/résumé is in a language other than English,
% then you probably want to change these so that when you
% copy-paste from the PDF or run pdftotext, the location
% and date marker icons for \cvevent will paste as correct
% translations. For example Spanish:
% \renewcommand{\locationname}{Ubicación}
% \renewcommand{\datename}{Fecha}


%% Use (and optionally edit if necessary) this .tex if you
%% want to use an author-year reference style like APA(6)
%% for your publication list
% \input{pubs-authoryear.tex}

%% Use (and optionally edit if necessary) this .tex if you
%% want an originally numerical reference style like IEEE
%% for your publication list
\input{pubs-num.tex}

%% sample.bib contains your publications
\addbibresource{sample.bib}

\begin{document}
\name{Carson S. Fischl}
\tagline{}
%% You can add multiple photos on the left or right
\photoR{2.8cm}{Globe_High}
% \photoL{2.5cm}{Yacht_High,Suitcase_High}

\personalinfo{%
  % Not all of these are required!
  \email{fischlcarson@gmail.com}
  \phone{+33 6 18 36 99 97}
  \mailaddress{29 Avenue Abadie}
  \location{Bordeaux, Gironde, France}
  \homepage{carsonfischl.vercel.app}
  \linkedin{carsonfischl}
  \github{carsonfischl}
  %% You can add your own arbitrary detail with
  %% \printinfo{symbol}{detail}[optional hyperlink prefix]
  % \printinfo{\faPaw}{Hey ho!}[https://example.com/]

  %% Or you can declare your own field with
  %% \NewInfoFiled{fieldname}{symbol}[optional hyperlink prefix] and use it:
  % \NewInfoField{gitlab}{\faGitlab}[https://gitlab.com/]
  % \gitlab{your_id}
  %%
  %% For services and platforms like Mastodon where there isn't a
  %% straightforward relation between the user ID/nickname and the hyperlink,
  %% you can use \printinfo directly e.g.
  % \printinfo{\faMastodon}{@username@instace}[https://instance.url/@username]
  %% But if you absolutely want to create new dedicated info fields for
  %% such platforms, then use \NewInfoField* with a star:
  % \NewInfoField*{mastodon}{\faMastodon}
  %% then you can use \mastodon, with TWO arguments where the 2nd argument is
  %% the full hyperlink.
  % \mastodon{@username@instance}{https://instance.url/@username}
}

\makecvheader
%% Depending on your tastes, you may want to make fonts of itemize environments slightly smaller
% \AtBeginEnvironment{itemize}{\small}

%% Set the left/right column width ratio to 6:4.
\columnratio{0.6}

% Start a 2-column paracol. Both the left and right columns will automatically
% break across pages if things get too long.
\begin{paracol}{2}
\cvsection{Expérience}

\cvevent{Représentant Étudiant}{IAE Bordeaux}{Septembre 2023 -- Présent}{Bordeaux, France}
\begin{itemize}
\item Représentant étudiant de l'université en classe et lors d'événements parascolaires.
\item Géstion avec succès la planification simultanée des événements étudiants, le support informatique pour le personnel universitaire et le déploiement du nouveau système de suivi de la présence des étudiants, Edusign.
\end{itemize}

\divider

\cvevent{Ingénieur Logiciel}{Dell Technologies}{Janvier 2023 -- Septembre 2023}{Ottawa, Canada}
\begin{itemize}
\item Développement d'un cadre de test multitfil utilisant Python pour la partie DU de l'infrastructure 5G de Dell.
\item Ajout de rapports LCOV, de la fonctionnalité pause/reprise TTI et d'autres améliorations à un pipeline devops complexe utilisant les intégrations Jenkins et Github
\end{itemize}

\divider

\cvevent{Stagiaire Génie DDR PHY}{Synopsys}{Octobre 2021 -- Août 2022}{Ottawa, Canada}
\begin{itemize}
\item Simulées et déboguées pour les puces de production les version de DDR PHY pour les clients industriel afin d'atténuer les risques techniques en production.
\item Ajout de rapports LCOV, de la fonctionnalité pause/reprise TTI et d'autres améliorations à un pipeline devops complexe utilisant les intégrations Jenkins et Github
\end{itemize}

\cvsection{Projets}

\cvevent{Collection de pièces anciennes}{\printinfo{\faArrowRight}{Lien}[https://comitas-collection.vercel.app/]}{Novembre 2022}{}
\begin{itemize}
\item Application full stack réalisée avec Next.js pour montrer ma collection personnelle de pièces rares et anciennes.
\item Photographies de haute qualité, JS/TS et le SSR inclus.
\end{itemize}

\divider

\cvevent{Site web pour une artiste locale}{\printinfo{\faArrowRight}{Lien}[https://abnlv.github.io/]}{Juin 2023}{}
%\begin{itemize}
%\item Application full stack réalisée avec Next.js pour montrer ma collection personnelle de pièces rares et anciennes.
%\end{itemize}

\medskip

% use ONLY \newpage if you want to force a page break for
% ONLY the current column
\newpage

\switchcolumn

\cvsection{Centres d'Interêts}

\cvtag{Histoire}
\cvtag{Naturalisme}
\cvtag{Haltérophilie}\\
\cvtag{Voyage en sac à dos}
\cvtag{La Course}


\cvsection{Prix et Distinctions}

\cvachievement{\faTrophy}{Bourse France Excellence}{Un de dix récipients Canadien}


\cvsection{Points Forts}

\cvtag{Analyse des Données}
\cvtag{Administration}\\
\cvtag{Présentations}
\cvtag{Développement d'Entreprise}

\divider\smallskip

\cvtag{React}
\cvtag{Next.js}
\cvtag{Python}\\
\cvtag{JIRA}
\cvtag{SQL}
\cvtag{R}\\
\cvtag{Excel}
\cvtag{Powerpoint}
\cvtag{Git}

\cvsection{Langues}

\cvskill{English}{5}
\divider

\cvskill{Français}{3.5}

%% Yeah I didn't spend too much time making all the
%% spacing consistent... sorry. Use \smallskip, \medskip,
%% \bigskip, \vspace etc to make adjustments.
\medskip

\cvsection{Formation}

\cvevent{MBA\ Administration d'Entreprise}{IAE Bordeaux - Université de Bordeaux}{Septembre 2023 -- Présent}{}

\divider

\cvevent{BCS\ Informatique}{Université Carleton}{Septembre 2020 -- Decembre 2022}{}

\divider

\cvevent{BSc\ Biochimie}{Université Carleton}{Septembre 2015 -- Août 2020}{}


\end{paracol}


\end{document}